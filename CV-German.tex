%% start of file `template.tex'.
%% Copyright 2006-2013 Xavier Danaux (xdanaux@gmail.com).
%
% This work may be distributed and/or modified under the
% conditions of the LaTeX Project Public License version 1.3c,
% available at http://www.latex-project.org/lppl/.


\documentclass[11pt,a4paper,sans]{moderncv}        % possible options include font size ('10pt', '11pt' and '12pt'), paper size ('a4paper', 'letterpaper', 'a5paper', 'legalpaper', 'executivepaper' and 'landscape') and font family ('sans' and 'roman')




% moderncv themes
\moderncvstyle{classic}                             % style options are 'casual' (default), 'classic', 'oldstyle' and 'banking'
\moderncvcolor{blue}                               % color options 'blue' (default), 'orange', 'green', 'red', 'purple', 'grey' and 'black'
%\renewcommand{\familydefault}{\sfdefault}         % to set the default font; use '\sfdefault' for the default sans serif font, '\rmdefault' for the default roman one, or any tex font name
%\nopagenumbers{}                                  % uncomment to suppress automatic page numbering for CVs longer than one page

\makeatletter
\renewcommand{\section}[1]
{ \vspace*{2.5ex \@plus 1ex \@minus .2ex}%
\phantomsection{}% reset the anchor for hyperrefs


\addcontentsline{toc}{part}{#1}%
\parbox[m]{\hintscolumnwidth}{\raggedleft  \hintfont{\color{color1}\rule{\hintscolumnwidth}{1pt}}}%<- auf 11pt geaendert
\hspace{\separatorcolumnwidth}%
\parbox[m]{\maincolumnwidth}{\sectionstyle{#1}}\\[1ex]}
\makeatother


% character encoding
%\usepackage[utf8]{inputenc}                       % if you are not using xelatex ou lualatex, replace by the encoding you are using
%\usepackage{CJKutf8}                              % if you need to use CJK to typeset your resume in Chinese, Japanese or Korean

%Deutsche Umlaute
\usepackage[utf8]{inputenc}
%\usepackage{showframe}

%Seitennummerierung
\usepackage{lastpage}
\rfoot{Seite \thepage\ von \pageref{LastPage}}

% adjust the page margins
\usepackage[scale=0.8]{geometry}
\setlength{\hintscolumnwidth}{3cm}                % if you want to change the width of the column with the dates
\setlength{\makecvheadnamewidth}{10cm}           % for the 'classic' style, if you want to force the width allocated to your name and avoid line breaks. be careful though, the length is normally calculated to avoid any overlap with your personal info; use this at your own typographical risks..

% personal data
\name{Martin}{Domnick}
\title{Lebenslauf}                               % optional, remove / comment the line if not wanted
\address{Alsterdorfer Str. 337}{22297 Hamburg}{}% optional, remove / comment the line if not wanted; the "postcode city" and "country" arguments can be omitted or provided empty
\phone[mobile]{01745982043}                   % optional, remove / comment the line if not wanted; the optional "type" of the phone can be "mobile" (default), "fixed" or "fax"
%\phone[fixed]{+2~(345)~678~901}
%\phone[fax]{+3~(456)~789~012}
\email{martin\_domnick@web.de}                               % optional, remove / comment the line if not wanted
%\homepage{www.johndoe.com}                         % optional, remove / comment the line if not wanted
%\social[linkedin]{john.doe}                        % optional, remove / comment the line if not wanted
%\social[twitter]{jdoe}                             % optional, remove / comment the line if not wanted
%\social[github]{jdoe}                              % optional, remove / comment the line if not wanted
\extrainfo{Geburtsdatum: 14.01.1987}                 % optional, remove / comment the line if not wanted
\photo[112pt][0.4pt]{picture4}                       % optional, remove / comment the line if not wanted; '64pt' is the height the picture must be resized to, 0.4pt is the thickness of the frame around it (put it to 0pt for no frame) and 'picture' is the name of the picture file
%\quote{Some quote}                                 % optional, remove / comment the line if not wanted




% to show numerical labels in the bibliography (default is to show no labels); only useful if you make citations in your resume
%\makeatletter
%\renewcommand*{\bibliographyitemlabel}{\@biblabel{\arabic{enumiv}}}
%\makeatother
%\renewcommand*{\bibliographyitemlabel}{[\arabic{enumiv}]}% CONSIDER REPLACING THE ABOVE BY THIS

% bibliography with mutiple entries
%\usepackage{multibib}
%\newcites{book,misc}{{Books},{Others}}
%----------------------------------------------------------------------------------
%            content
%----------------------------------------------------------------------------------
\begin{document}
%\begin{CJK*}{UTF8}{gbsn}                          % to typeset your resume in Chinese using CJK
%-----       resume       ---------------------------------------------------------
\makecvtitle

\section{Bildung}
\vspace{5pt}

\cventry{2007 -- 2011}{Diplominformatiker}{Hochschule für Technik und Wirtschaft}{Dresden}{Abschlussnote 1.8}{}  % arguments 3 to 6 can be left empty
\cventry{2006 -- 2007}{Fachhochschulreife}{Europäische
Wirtschafts- und Sprachenakademie}{Dresden}{}{}
\cventry{2004 -- 2006}{Staatlich geprüfter Wirtschaftsassistent}{Berufsfachschule für Wirtschaft der Weiterbildungsakademie Dresden gGmbH}{Dresden}{}{}

\section{Diplomarbeit}
\vspace{5pt}

\cvitem{Titel}{\emph{Konzeption einer grafischen Nutzeroberfläche zur prozessorientierten Modellierung
und Simulation}}
\cvitem{Beschreibung}{Entwicklung einer Erweiterung für den WPF Designer {\textit Cider}  (Visual Studio 2010) für die grafische Erstellung von Simulationsmodellen}
\cvitem{Note}{1.3}

\section{Praxis}
\vspace{-5pt}

\subsection{Berufserfahrung}
\cventry{2015 -- Heute}{Senior Softwareentwickler}{Statista GmbH}{Hamburg}{}{Entwicklung von Anwendungs-Software und Backend-Diensten (Java, .Net)}
\cventry{2012 -- 2015}{Softwareentwickler}{AIS Automation Dresden GmbH}{Dresden}{}{Entwurf, Spezifikation und Implementierung von Softwarelösungen für den industriellen Bereich (.Net)}
\subsection{Studentische Hilfskraft}
\cventry{2010 -- 2011}{Softwareentwickler}{Hochschule für Technik und Wirtschaft}{Dresden}{}{Erweiterung für bestehenden
Softwarelösungen der Hochschule (.Net)}\pagebreak
\subsection{Praktikum}
\cventry{September 2009 -- Februar 2010}{Softwareentwickler}{CES IT-Systemhaus GmbH}{Dresden}{Weiterentwicklung einer bestehenden Immobilienverwaltungssoftware (.Net)}{}

\section{Fremdsprachenkenntnisse}
\cvitemwithcomment{Englisch}{Fließend}{}
\cvitemwithcomment{Italienisch}{Erweiterte Grundkenntnisse}{Erfolgreicher Abschluss eines Sprachstudiums an der {\textit Università per Stranieri di Perugia} (April 2015 - Juni 2015) auf der
Sprachniveaustufe {\textit B1} des Gemeinsamen Europäischen Referenzrahmens}{}

\section{Technologien}
\vspace{-5pt}


\cvitem{Programming}{Java (Spring-Boot, Hibernate), .Net (C\#, WPF, Windows-Forms, Silverlight), HTML, SQL, XML, JSON, Kotlin}
\cvitem{Design-Pattern}{MVVM, Inversion of Control, Dependency Injection, REST}
\cvitem{Cloud}{AWS (SQS, S3, Lambda, EC2, SES)}
\cvitem{Tools}{Visual Studio, Intelijii, Git, Bash}
\cvitem{Unit-Tests}{Visual Studio Testtools, NUnit, JUnit, Mocking-Frameworks}
\cvitem{Devops}{Linux (Debian/CentOS), Windows (Desktop + Server) }
\cvitem{Office}{Excel, Word, Power Point}


\section{Links}
\cvitem{Xing}{\textcolor{red}{\underline{\href{https://www.xing.com/profile/Martin_Domnick3}{Online Profil Öffnen}}}}

% Publications from a BibTeX file without multibib
%  for numerical labels: \renewcommand{\bibliographyitemlabel}{\@biblabel{\arabic{enumiv}}}% CONSIDER MERGING WITH PREAMBLE PART
%  to redefine the heading string ("Publications"): \renewcommand{\refname}{Articles}
%\nocite{*}
%\bibliographystyle{plain}
%\bibliography{publications}                        % 'publications' is the name of a BibTeX file

% Publications from a BibTeX file using the multibib package
%\section{Publications}
%\nocitebook{book1,book2}
%\bibliographystylebook{plain}
%\bibliographybook{publications}                   % 'publications' is the name of a BibTeX file
%\nocitemisc{misc1,misc2,misc3}
%\bibliographystylemisc{plain}
%\bibliographymisc{publications}                   % 'publications' is the name of a BibTeX file

\clearpage

%\clearpage\end{CJK*}                              % if you are typesetting your resume in Chinese using CJK; the \clearpage is required for fancyhdr to work correctly with CJK, though it kills the page numbering by making \lastpage undefined
\end{document}


%% end of file `template.tex'.
